\documentclass[margin,line]{res}

\usepackage[xetex,%
         plainpages=false,%
%		 hidelinks=true,%
        pdfstartview=FitH]{hyperref}

\usepackage[CJKchecksingle]{xeCJK}

\usepackage{palatino}
\usepackage{hologo}

%\usepackage[T1]{fontenc}
%\usepackage[sc]{mathpazo}

\linespread{1.05} 

%\XeTeXlinebreaklocale "zh"
%\XeTeXlinebreakskip = 0pt plus 1pt

\setCJKmainfont[BoldFont={FZSongHei-B07}]{FZBeiWeiKaiShu-S19}
%\setCJKmainfont[BoldFont={FZCuSong-B09}]{FZBaoSong-Z04}
\setCJKsansfont[BoldFont={FZChaoCuHei-M10}]{FZHei-B01}
\setCJKmonofont[BoldFont={FZCuYuan-M03}]{FZZhunYuan-M02}
\setCJKfamilyfont{name}[BoldFont={FZLiShu-S01}]{FZLiShu-S01}
\punctstyle{kaiming}

\oddsidemargin -.5in
\evensidemargin -.5in
\textwidth=6.0in
\itemsep=0in
\parsep=0in


\newenvironment{list1}{
  \begin{list}{\ding{113}}{%
      \setlength{\itemsep}{0in}
      \setlength{\parsep}{0in} \setlength{\parskip}{0in}
      \setlength{\topsep}{0in} \setlength{\partopsep}{0in} 
      \setlength{\leftmargin}{0.17in}}}{\end{list}}
\newenvironment{list2}{
  \begin{list}{$\bullet$}{%
      \setlength{\itemsep}{0in}
      \setlength{\parsep}{0in} \setlength{\parskip}{0in}
      \setlength{\topsep}{0in} \setlength{\partopsep}{0in} 
      \setlength{\leftmargin}{0.2in}}}{\end{list}}

\newcommand{\http}{http:/\hspace{-0.3ex}/}
\newcommand{\hindent}{\mbox{\hspace{8ex}}}

\begin{document}

\name{\Large\CJKfamily{name} 王旭 \vspace*{.05in}}

\begin{resume}
\section{联系信息}
\vspace{.05in}
\begin{tabular}{@{}p{2in}|p{3in}}
北京市,海淀区 	   & \textsf{电话:} {13911635527}\\         
蓟门里东2楼104     & \textsf{邮箱:} {gnawux@gmail.com}\\       
邮政编码: 100088   & \textsf{主页:} {\http{}wangxu.me/blog/}\\
\end{tabular}

\section{个人介绍}

接近4年的Hadoop HDFS开发经验,中国移动研究院大云团队核心开发成员,负责NameNode元数据管理和DataNode磁盘管理相关等方面的开发项目,与Hadoop开源社区中的国内外开发者保持密切联系。超过10年的Linux使用与维护经验,熟练使用C/C++, Java等多种开发语言,Python, Perl等脚本语言及各种Unix工具进行开发与日常维护工作。

\section{工作意向}

位于北京的\textsc{ Apache Hadoop}平台相关的开发与维护工作,主要是HDFS相关的工作;或分布存储系统相关的研发工作。

\section{当前工作}
\textsc{Hadoop }\textsf{相关工作}\\
\vspace*{-.1in}
\begin{list1}
\item[] 中国移动研究院研究员/项目经理(2007年7月至今),负责{\sc Hadoop HDFS}相关开发维护工作以及开源社区交流工作,包括Hadoop的评估测试、问题修复、功能改进等(Apache JIRA ID \href{https://issues.apache.org/jira/secure/ViewProfile.jspa?name=gnawux}{gnawux}):
\begin{list2}
\vspace*{.05in}
\item 2010-2011年,为Trunk提交磁盘刷新功能补丁\href{https://issues.apache.org/jira/browse/HDFS-1362}{\tt HDFS-1362};
\item 2008年,提交补丁\href{https://issues.apache.org/jira/browse/HADOOP-4742}{\tt HADOOP-4742},修复Hadoop 0.18丢失数据块副本的严重Bug;
\item 2008年4月至今,维护中国移动研究院的 Hadoop 分支,包含多NameNode同步补丁;\\
          \hindent\href{http://github.com/gnawux/hadoop-cmri}{\tt\http{}github.com/gnawux/hadoop-cmri}
\item 2008-2009年,开发并维护中国移动研究院的 Hadoop 测试工具(Python);\\
          \hindent\href{http://code.google.com/p/hadoop-test/}{\tt\http{}code.google.com/p/hadoop-test/} 
\item 2010年9月,参加Hadoop in China,介绍中国移动研究院的Hadoop开发情况;\\
          \hindent\href{http://wangxu.me/blog/p/445}{\tt \http{}wangxu.me/blog/p/445}
\item 2009 年 11 月参加 \href{http://wiki.apache.org/apachecon/ApacheRoadshowAsia09Beijing}{Apache Roadshow Asia 2009},介绍中国移动研究院在 HDFS 方面的工作;
\item 开发工作部分记录:\enskip {\href{http://gnawux.info/hadoop/}{\tt \http{}gnawux.info/hadoop/}}
\end{list2}
\end{list1}

\textsf{其他工作}\\
\vspace*{-.1in}
\begin{list2}
\item 【Linux集群管理】 \enskip 2010年至今,中国移动通信研究院大云平台系统管理员,负责系统配置维护、性能优化,涉及到虚拟化平台和Hadoop平台。
\item 【云存储】 \enskip 2010年至今, 项目经理与设计人员,负责中国移动研究院的云存储项目。
\item 【书】 \enskip 负责中国移动研究院云计算技术相关图书编写工作。
\item 【译】 \enskip 业余时间翻译{\it Cassandra: The Definitive Guide}\/一书,预计将于2011年4月完成,由图灵图书出版。
\end{list2}

\section{过往工作}

\textsc{Linux}\textsf{及其他开源相关工作} \\
\vspace*{-.1in}
\begin{list2}
\item 【书】\enskip \href{http://wangxu.me/blog/p/tag/unleashed}{《Debian 标准教程》}, 2004-2008年持续更新、完善,最终于2009年9月由人民邮电出版社出版。
\item 【活动】\enskip \href{http://linuxfb.org}{LinuxFB/北邮 BBS Linux 每月聚会活动}联合发起人和组织者。
\item 【文章】\enskip 常年在业余时间写/译关于 Linux Kernel, 文件系统,虚拟化, NoSQL 等内容的技术文章,发于 \href{http://wangxu.me/blog/}{个人网站 \http{}wanxu.me/blog/},其中关于 Zookeeper 的译文发于《程序员》 2010 年第 2 期。
\item 【\LaTeX】\enskip 2007 年制作并发布了\href{http://code.google.com/p/latex-bupt}{北邮研究生毕业论文 \LaTeX 模板第一版}的代码、样例和说明。
\item 【编辑】\enskip 2004年曾担任 LinuxFocus (随 Linux 文档项目 (LDP) 发布的一份自由多语言杂志) 志愿编辑,负责管理中文部分的页面,同时还负责翻译和撰写技术文章。
\end{list2}

\textsf{培训讲师} \hfill \textsf{2005 年至 2007 年}\\
在北京、广东、武汉、天津等地为多家运营商、开发商进行 Linux 和 3G (WCDMA/TD-SCDMA) 的核心网、业务及无线接入网相关员工培训和认证培训。

\textsf{软件开发人员}\hfill \textsf{2002 年至 2004 年}\\
\vspace*{-.1in}
\begin{list1}
\item[]  参与星河亮点通信软件有限公司 PHS 路测软件开发(Visual C++,核心开发人员)和移动Agent相关国家项目(Java,项目组长)。
\end{list1}


\section{教育背景}
\textsf{北京邮电大学}\\
\vspace*{-.1in}
\begin{list1}
\item[] 电信工程学院, 工学博士, 2002 年 9月至 2007 年 7 月
\begin{list2}
\vspace*{.05in}
\item 导师: 张平\ 教授
\end{list2}
\end{list1}

\textsf{北京工业大学}\\
\vspace*{-.1in}
\begin{list1}
\item[] 电子工程系, 工学学士, 1998 年 9月至 2002 年 7 月 
\end{list1}

\section{参考}
\textsf{部分技术文章}\\
\vspace*{-.1in}
\begin{list1}
\item[] Hadoop 相关技术文章(\href{http://wangxu.me/blog/p/tag/hadoop}{\tt\http{}wangxu.me/blog/p/tag/hadoop})
\begin{list2}
\vspace*{.05in}
\item DataNode的磁盘管理以及HDFS-1362: \href{http://wangxu.me/blog/p/516}{\tt\http{}wangxu.me/blog/p/516}
\item HDFS退服节点的方法: \href{http://labs.chinamobile.com/mblog/225_18378}{\tt\http{}labs.chinamobile.com/mblog/225\_18378}
\item dfs.datanode.max.xcievers: hadoop 的一个无文档的参数: \\
	\hindent\href{http://labs.chinamobile.com/mblog/225_18094}{\tt\http{}labs.chinamobile.com/mblog/225\_18094}
\vspace*{.05in}
\end{list2}
\item[] 系统管理与版本管理相关文章,多数是管理过程中的问题分析处理与案例和脚本记录:
\begin{list2}
\vspace*{.05in}
\item 用SVM同步SVN仓库及解决锁问题: \href{http://wangxu.me/blog/p/523}{\tt\http{}wangxu.me/blog/p/523}
\item 同时操作多台主机的命令行: \href{http://wangxu.me/blog/p/166}{\tt\http{}wangxu.me/blog/p/166}
\item 一行的日志分析脚本: \href{http://wangxu.me/blog/p/145}{\tt\http{}//wangxu.me/blog/p/145}
\vspace*{.05in}
\end{list2}
\end{list1}

\textsf{部分翻译文章}\\
\vspace*{-.1in}
\begin{list1}
\item[] NoSQL、Hadoop 与 Java 相关技术文章翻译:
\begin{list2}
\vspace*{.05in}
\item Cassandra 实例 \href{http://wangxu.me/blog/p/383}{\tt\http{}wangxu.me/blog/p/383}
\item Observers: 让ZooKeeper更具可伸缩性: \href{http://wangxu.me/blog/p/320}{\tt\http{}wangxu.me/blog/p/320}
\item Avro: 大数据的数据格式: \href{http://wangxu.me/blog/p/279}{\tt\http{}wangxu.me/blog/p/279}
\item Java SE 6 Hotspot${\,}^{TM}$ 虚拟机垃圾回收调优: \href{http://wangxu.me/blog/p/209}{\tt\http{}wangxu.me/blog/p/209}
\vspace*{.05in}
\end{list2}
\item[] Linux 内核与图形相关以及其他软件相关技术文章翻译:
\begin{list2}
\vspace*{.05in}
\item LWN: 空指针的乐趣(1): \href{http://wangxu.me/blog/p/39}{\tt\http{}wangxu.me/blog/p/39}
\item KernelTrap: Linux: The Journaling Block Device: \href{http://wangxu.me/blog/p/151}{\tt\http{}wangxu.me/blog/p/151}
\item LWN: GEM vs. TTM: \href{http://wangxu.me/blog/p/124}{\tt\http{}wangxu.me/blog/p/124}
\item LWN: What is RCU, Fundamentally?: \href{http://wangxu.me/blog/p/122}{\tt\http{}wangxu.me/blog/p/122}
\end{list2}
\end{list1}
{\small \textsf{注:所有译文的英文原文都来自于互联网,并在开头明确注明原文发布时间、链接、作者和翻译时间,包括LWN、Cloudera博客等在内的大部分文章在翻译时取得了发布者的授权。}}

\end{resume}

\vfill
\vfill
\mbox{ }\hspace{100pt}	\small\Resume\ Generated with \hologo{XeLaTeX}

\end{document}

